\documentclass[spanish]{article}
\usepackage[T1]{fontenc}
\usepackage[utf8]{inputenc}
\usepackage{babel}
\usepackage{graphicx}
\usepackage{array}
\usepackage{float}
\usepackage[margin=2.5cm]{geometry}


% Para eliminar el título "references" de la bibliografía
\usepackage{etoolbox}
\patchcmd{\thebibliography}{\section*{\refname}}{}{}{}
\begin{document}

	\title{Propuesta\\Anteproyecto de título}
	\maketitle

\section{Identificación}
\subsection{Estudiantes}
\begin{itemize}
		\item NOMBRE : Martin Araneda Acuña
		\item DIRECCIÓN : Psje Veintidós, \#85, La Floresta IV, Hualpén
		\item TELÉFONO : +56983828885
        \item CARRERA : Ingeniería Civil en Informática
		\item E-MAIL : martin.araneda1501@alumnos.ubiobio.cl
\end{itemize}
\vspace{2mm}
\begin{itemize}
    \item NOMBRE : Christopher Cromer
    \item DIRECCIÓN : Roberto Matta 204, Departamento 625, Concepción
    \item TELÉFONO : +56990864256
    \item CARRERA : Ingeniería Civil en Informática
    \item E-MAIL : christopher.cromer1501@alumnos.ubiobio.cl
\end{itemize}

\subsection{Profesor Guía}
	\begin{itemize}
		\item NOMBRE : Clemente Rubio
		\item E-MAIL : clrubio@ubiobio.cl
	\end{itemize}

\subsection{Personas, Instituciones O Empresas En Que Se Solicitará Apoyo Y Asesoría}
	\begin{itemize}
		\item NOMBRE: Clemente Rubio
		\item Rubro:
		\item Email: clrubio@ubiobio.cl
		\item Firma:
	\end{itemize}

\subsection{Nombre De La Persona Responsable De La Empresa Que Supervisara Al Alumno}
	\begin{itemize}
		\item NOMBRE: Clemente Rubio
		\item Cargo:
		\item Email: clrubio@ubiobio.cl
	\end{itemize}

\section{Título Anteproyecto}

Implementación de una inteligencia artificial en video juegos utilizando un lenguaje lógico de programación

\section{Descripción del Problema}

Se va utilizar un lenguaje lógico de programación para poner en funcionamiento una inteligencia artificial autónoma\
desarrollado en un motor de video juegos.

\section{Objetivos de la Actividad}
\subsection{Objetivo General:}

La finalidad de esta actividad de titulación es el desarrollo de un lenguaje de programación de tipo Prolog para poder\
implementar una inteligencia artificial que permita evitar ciertos obstáculos.

\subsection{Objetivos Específicos:}

\begin{enumerate}
    \item Revisar bibliografía sobre Prolog, el motor Godot y programación de video juegos.
    \item Analizar la información recopilado de la bibliografía investigada.
    \item Crear el lenguaje de programación tipo Prolog.
    \item Implementar el lenguaje de programación en el motor Godot.
    \item Desarrollar un video juego usando inteligencia artificial basado en el lenguaje tipo Prolog.
\end{enumerate}

\section{Descripción de las actividades (Plan de trabajo)}

[En la siguiente sección se describe las actividades que se utilizarán para realizar la investigación o desarrollo de SW. Es decir las actividades para cumplir cada uno de los objetivos especificos]

[no verbos, y puede haber una actividad por objetivo o mas por objetivo]

\begin{itemize}
    \item ...
    \item ...
\end{itemize}


\section{Justificación del Proyecto}

El beneficio de usar un lenguaje lógico en vez de funcional es poder programar una inteligencia artificial que tome decisiones\
de la\
misma forma que una persona real piensa usando datos basado en el entorno.

Es necesario para así simular de manera mas realista el comportamiento humano de una inteligencia artificial y poder ser\
adaptado a otros tipos de juegos y motores.

\section{Análisis de los Principales Trabajos Realizados en el área o tema de la propuesta }

[No se encuentra trabajo previo en la Universidad del Bío-Bío, pero sí en otras universidades de Latinoamérica o empresas].
[Si los hay, describir en qué se diferencian, cuál es el aporte de este proyecto)]

	\begin{enumerate}
		\item Video Game ....
		\item Aprendizado .....

	\end{enumerate}

	También, hay trabajos publicados en revistas de investigación que se relacionan con el trabajo a realizar, estos serían los siguientes:
	\begin{enumerate}
		\item En el artículo ....
		\item En la pagina ...

	\end{enumerate}

	En la siguiente sección se describe el ambiente de hardware y software que se utiliza para la implementación y ejecución de los algoritmos a comparar.



\section{Resultados Esperados de la investigación (INV) o Descripción del ambiente de Software esperado (SW)}

Esencialmente se espera que un agente en el video juego pueda evitar obstáculos a través de la toma de decisiones utilizando\
la inteligencia artificial implementada y así llega a la meta.

\section{Planificación del trabajo a desarrollar: Carta Gantt}

	En esta sección se presenta la carta gantt del plan de trabajo a desarrollar para la investigación

\begin{table}[H]
\centering
%\resizebox{\textwidth}{!}{%
\begin{tabular}{|l|l|l|}
\hline
\textbf{Actividad}      & \textbf{Duración} & \textbf{I/F} \\ \hline
Actividad 1             & 1 mes            &  sept\\ \hline
Actividad 1             & 1 mes            &  sept-oct\\ \hline
Actividad 1             & 1 mes            &  \\ \hline
Actividad 1             & 1 mes             & \\ \hline
\end{tabular}%
%}
\end{table}

\section{Referencias}
\bibliographystyle{ieeetr}
\bibliography{references.bib}
\end{document}